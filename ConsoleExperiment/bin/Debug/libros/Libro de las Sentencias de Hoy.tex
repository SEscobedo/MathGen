\documentclass[12pt]{book}
\usepackage[spanish]{babel}
\usepackage{amsmath}
\usepackage[utf8]{inputenc}
\newtheorem{theorem}{Teorema}[chapter]
\newtheorem{definition}{Definición}[chapter]
\newtheorem{axiom}{Axioma}[chapter]
\newtheorem{proof}{Demostración}
\title{Libro de las Sentencias de Hoy}
\author{Yo Merengues}
\date{}
\begin{document}
\maketitle
 Introducimos sin definición los conceptos siguientes: \textit{
Pedro, hombre, animal, vivo, Juan, tigre, sustancia, Dios, mortal, reptil, lagarto, cocodrilo, rico}
\\
\\Mediante estos conceptos formamos los axiomas que siguen. \\
\textbf{Axiomas:}
\begin{enumerate}
\item Pedro es hombre
\item todo hombre es animal
\item todo animal es vivo
\item Juan no es hombre
\item todo tigre es animal
\item todo vivo es sustancia
\item Dios no es mortal
\item todo vivo es mortal
\item todo reptil es animal
\item todo lagarto es reptil
\item todo cocodrilo es reptil
\item todo hombre es rico
\end{enumerate}De la lista anterior se deducen los próximos teoremas. \\

\begin{theorem}
algún animal es Pedro
\label{th: 12}
\end{theorem}\begin{proof}\\Pedro es hombre	 (Premisa mayor) \\todo hombre es animal	 (Premisa menor) \\Luego, algún animal es Pedro	FigIV:Bramantip \\ \end{proof}
\begin{theorem}
Juan no es Pedro
\label{th: 13}
\end{theorem}\begin{proof}\\Pedro es hombre	 (Premisa mayor) \\Juan no es hombre	 (Premisa menor) \\Luego, Juan no es Pedro	FigII:Camestres \\ \end{proof}
\begin{theorem}
algún rico es Pedro
\label{th: 14}
\end{theorem}\begin{proof}\\Pedro es hombre	 (Premisa mayor) \\todo hombre es rico	 (Premisa menor) \\Luego, algún rico es Pedro	FigIV:Bramantip \\ \end{proof}
\begin{theorem}
Pedro es animal
\label{th: 15}
\end{theorem}\begin{proof}\\todo hombre es animal	 (Premisa mayor) \\Pedro es hombre	 (Premisa menor) \\Luego, Pedro es animal	FigI:Barbara \\ \end{proof}
\begin{theorem}
algún vivo es hombre
\label{th: 16}
\end{theorem}\begin{proof}\\todo hombre es animal	 (Premisa mayor) \\todo animal es vivo	 (Premisa menor) \\Luego, algún vivo es hombre	FigIV:Bramantip \\ \end{proof}
\begin{theorem}
algún rico es animal
\label{th: 17}
\end{theorem}\begin{proof}\\todo hombre es animal	 (Premisa mayor) \\todo hombre es rico	 (Premisa menor) \\Luego, algún rico es animal	FigIII:Darapti \\ \end{proof}
\begin{theorem}
todo hombre es vivo
\label{th: 18}
\end{theorem}\begin{proof}\\todo animal es vivo	 (Premisa mayor) \\todo hombre es animal	 (Premisa menor) \\Luego, todo hombre es vivo	FigI:Barbara \\ \end{proof}
\begin{theorem}
todo tigre es vivo
\label{th: 19}
\end{theorem}\begin{proof}\\todo animal es vivo	 (Premisa mayor) \\todo tigre es animal	 (Premisa menor) \\Luego, todo tigre es vivo	FigI:Barbara \\ \end{proof}
\begin{theorem}
algún sustancia es animal
\label{th: 20}
\end{theorem}\begin{proof}\\todo animal es vivo	 (Premisa mayor) \\todo vivo es sustancia	 (Premisa menor) \\Luego, algún sustancia es animal	FigIV:Bramantip \\ \end{proof}
\begin{theorem}
algún mortal es animal
\label{th: 21}
\end{theorem}\begin{proof}\\todo animal es vivo	 (Premisa mayor) \\todo vivo es mortal	 (Premisa menor) \\Luego, algún mortal es animal	FigIV:Bramantip \\ \end{proof}
\begin{theorem}
todo reptil es vivo
\label{th: 22}
\end{theorem}\begin{proof}\\todo animal es vivo	 (Premisa mayor) \\todo reptil es animal	 (Premisa menor) \\Luego, todo reptil es vivo	FigI:Barbara \\ \end{proof}
\begin{theorem}
Pedro no es Juan
\label{th: 23}
\end{theorem}\begin{proof}\\Juan no es hombre	 (Premisa mayor) \\Pedro es hombre	 (Premisa menor) \\Luego, Pedro no es Juan	FigII:Cesare \\ \end{proof}
\begin{theorem}
algún animal no es Juan
\label{th: 24}
\end{theorem}\begin{proof}\\Juan no es hombre	 (Premisa mayor) \\todo hombre es animal	 (Premisa menor) \\Luego, algún animal no es Juan	FigIV:Fesapo \\ \end{proof}
\begin{theorem}
algún rico no es Juan
\label{th: 25}
\end{theorem}\begin{proof}\\Juan no es hombre	 (Premisa mayor) \\todo hombre es rico	 (Premisa menor) \\Luego, algún rico no es Juan	FigIV:Fesapo \\ \end{proof}
\begin{theorem}
algún vivo es tigre
\label{th: 26}
\end{theorem}\begin{proof}\\todo tigre es animal	 (Premisa mayor) \\todo animal es vivo	 (Premisa menor) \\Luego, algún vivo es tigre	FigIV:Bramantip \\ \end{proof}
\begin{theorem}
todo animal es sustancia
\label{th: 27}
\end{theorem}\begin{proof}\\todo vivo es sustancia	 (Premisa mayor) \\todo animal es vivo	 (Premisa menor) \\Luego, todo animal es sustancia	FigI:Barbara \\ \end{proof}
\begin{theorem}
algún mortal es sustancia
\label{th: 28}
\end{theorem}\begin{proof}\\todo vivo es sustancia	 (Premisa mayor) \\todo vivo es mortal	 (Premisa menor) \\Luego, algún mortal es sustancia	FigIII:Darapti \\ \end{proof}
\begin{theorem}
todo vivo no es Dios
\label{th: 29}
\end{theorem}\begin{proof}\\Dios no es mortal	 (Premisa mayor) \\todo vivo es mortal	 (Premisa menor) \\Luego, todo vivo no es Dios	FigII:Cesare \\ \end{proof}
\begin{theorem}
todo animal es mortal
\label{th: 30}
\end{theorem}\begin{proof}\\todo vivo es mortal	 (Premisa mayor) \\todo animal es vivo	 (Premisa menor) \\Luego, todo animal es mortal	FigI:Barbara \\ \end{proof}
\begin{theorem}
algún sustancia es mortal
\label{th: 31}
\end{theorem}\begin{proof}\\todo vivo es mortal	 (Premisa mayor) \\todo vivo es sustancia	 (Premisa menor) \\Luego, algún sustancia es mortal	FigIII:Darapti \\ \end{proof}
\begin{theorem}
Dios no es vivo
\label{th: 32}
\end{theorem}\begin{proof}\\todo vivo es mortal	 (Premisa mayor) \\Dios no es mortal	 (Premisa menor) \\Luego, Dios no es vivo	FigII:Camestres \\ \end{proof}
\begin{theorem}
algún vivo es reptil
\label{th: 33}
\end{theorem}\begin{proof}\\todo reptil es animal	 (Premisa mayor) \\todo animal es vivo	 (Premisa menor) \\Luego, algún vivo es reptil	FigIV:Bramantip \\ \end{proof}
\begin{theorem}
todo lagarto es animal
\label{th: 34}
\end{theorem}\begin{proof}\\todo reptil es animal	 (Premisa mayor) \\todo lagarto es reptil	 (Premisa menor) \\Luego, todo lagarto es animal	FigI:Barbara \\ \end{proof}
\begin{theorem}
todo cocodrilo es animal
\label{th: 35}
\end{theorem}\begin{proof}\\todo reptil es animal	 (Premisa mayor) \\todo cocodrilo es reptil	 (Premisa menor) \\Luego, todo cocodrilo es animal	FigI:Barbara \\ \end{proof}
\begin{theorem}
algún animal es lagarto
\label{th: 36}
\end{theorem}\begin{proof}\\todo lagarto es reptil	 (Premisa mayor) \\todo reptil es animal	 (Premisa menor) \\Luego, algún animal es lagarto	FigIV:Bramantip \\ \end{proof}
\begin{theorem}
algún animal es cocodrilo
\label{th: 37}
\end{theorem}\begin{proof}\\todo cocodrilo es reptil	 (Premisa mayor) \\todo reptil es animal	 (Premisa menor) \\Luego, algún animal es cocodrilo	FigIV:Bramantip \\ \end{proof}
\begin{theorem}
Pedro es rico
\label{th: 38}
\end{theorem}\begin{proof}\\todo hombre es rico	 (Premisa mayor) \\Pedro es hombre	 (Premisa menor) \\Luego, Pedro es rico	FigI:Barbara \\ \end{proof}
\begin{theorem}
algún animal es rico
\label{th: 39}
\end{theorem}\begin{proof}\\todo hombre es rico	 (Premisa mayor) \\todo hombre es animal	 (Premisa menor) \\Luego, algún animal es rico	FigIII:Darapti \\ \end{proof}
\begin{theorem}
algún animal es hombre
\label{th: 40}
\end{theorem}\begin{proof}\\Pedro es hombre	 (Premisa mayor) \\algún animal es Pedro	 (Premisa menor) \\Luego, algún animal es hombre	FigI:Darii \\ \end{proof}
\begin{theorem}
algún rico es hombre
\label{th: 41}
\end{theorem}\begin{proof}\\Pedro es hombre	 (Premisa mayor) \\algún rico es Pedro	 (Premisa menor) \\Luego, algún rico es hombre	FigI:Darii \\ \end{proof}
\begin{theorem}
algún vivo es Pedro
\label{th: 42}
\end{theorem}\begin{proof}\\Pedro es hombre	 (Premisa mayor) \\todo hombre es vivo	 (Premisa menor) \\Luego, algún vivo es Pedro	FigIV:Bramantip \\ \end{proof}
\begin{theorem}
algún vivo es animal
\label{th: 43}
\end{theorem}\begin{proof}\\todo hombre es animal	 (Premisa mayor) \\algún vivo es hombre	 (Premisa menor) \\Luego, algún vivo es animal	FigI:Darii \\ \end{proof}
\begin{theorem}
algún sustancia es hombre
\label{th: 44}
\end{theorem}\begin{proof}\\todo hombre es animal	 (Premisa mayor) \\todo animal es sustancia	 (Premisa menor) \\Luego, algún sustancia es hombre	FigIV:Bramantip \\ \end{proof}
\begin{theorem}
algún mortal es hombre
\label{th: 45}
\end{theorem}\begin{proof}\\todo hombre es animal	 (Premisa mayor) \\todo animal es mortal	 (Premisa menor) \\Luego, algún mortal es hombre	FigIV:Bramantip \\ \end{proof}
\begin{theorem}
Pedro es vivo
\label{th: 46}
\end{theorem}\begin{proof}\\todo animal es vivo	 (Premisa mayor) \\algún animal es Pedro	 (Premisa menor) \\Luego, Pedro es vivo	FigIII:Datisi \\ \end{proof}
\begin{theorem}
algún rico es vivo
\label{th: 47}
\end{theorem}\begin{proof}\\todo animal es vivo	 (Premisa mayor) \\algún rico es animal	 (Premisa menor) \\Luego, algún rico es vivo	FigI:Darii \\ \end{proof}
\begin{theorem}
algún sustancia es vivo
\label{th: 48}
\end{theorem}\begin{proof}\\todo animal es vivo	 (Premisa mayor) \\algún sustancia es animal	 (Premisa menor) \\Luego, algún sustancia es vivo	FigI:Darii \\ \end{proof}
\begin{theorem}
algún mortal es vivo
\label{th: 49}
\end{theorem}\begin{proof}\\todo animal es vivo	 (Premisa mayor) \\algún mortal es animal	 (Premisa menor) \\Luego, algún mortal es vivo	FigI:Darii \\ \end{proof}
\begin{theorem}
Dios no es animal
\label{th: 50}
\end{theorem}\begin{proof}\\todo animal es vivo	 (Premisa mayor) \\todo vivo no es Dios	 (Premisa menor) \\Luego, Dios no es animal	FigIV:Camenes \\ \end{proof}
\begin{theorem}
todo lagarto es vivo
\label{th: 51}
\end{theorem}\begin{proof}\\todo animal es vivo	 (Premisa mayor) \\todo lagarto es animal	 (Premisa menor) \\Luego, todo lagarto es vivo	FigI:Barbara \\ \end{proof}
\begin{theorem}
todo cocodrilo es vivo
\label{th: 52}
\end{theorem}\begin{proof}\\todo animal es vivo	 (Premisa mayor) \\todo cocodrilo es animal	 (Premisa menor) \\Luego, todo cocodrilo es vivo	FigI:Barbara \\ \end{proof}
\begin{theorem}
algún lagarto es vivo
\label{th: 53}
\end{theorem}\begin{proof}\\todo animal es vivo	 (Premisa mayor) \\algún animal es lagarto	 (Premisa menor) \\Luego, algún lagarto es vivo	FigIII:Datisi \\ \end{proof}
\begin{theorem}
algún cocodrilo es vivo
\label{th: 54}
\end{theorem}\begin{proof}\\todo animal es vivo	 (Premisa mayor) \\algún animal es cocodrilo	 (Premisa menor) \\Luego, algún cocodrilo es vivo	FigIII:Datisi \\ \end{proof}
\begin{theorem}
algún vivo no es Juan
\label{th: 55}
\end{theorem}\begin{proof}\\Juan no es hombre	 (Premisa mayor) \\algún vivo es hombre	 (Premisa menor) \\Luego, algún vivo no es Juan	FigII:Festino \\ \end{proof}
\begin{theorem}
algún sustancia es tigre
\label{th: 56}
\end{theorem}\begin{proof}\\todo tigre es animal	 (Premisa mayor) \\todo animal es sustancia	 (Premisa menor) \\Luego, algún sustancia es tigre	FigIV:Bramantip \\ \end{proof}
\begin{theorem}
algún mortal es tigre
\label{th: 57}
\end{theorem}\begin{proof}\\todo tigre es animal	 (Premisa mayor) \\todo animal es mortal	 (Premisa menor) \\Luego, algún mortal es tigre	FigIV:Bramantip \\ \end{proof}
\begin{theorem}
algún hombre es sustancia
\label{th: 58}
\end{theorem}\begin{proof}\\todo vivo es sustancia	 (Premisa mayor) \\algún vivo es hombre	 (Premisa menor) \\Luego, algún hombre es sustancia	FigIII:Datisi \\ \end{proof}
\begin{theorem}
todo hombre es sustancia
\label{th: 59}
\end{theorem}\begin{proof}\\todo vivo es sustancia	 (Premisa mayor) \\todo hombre es vivo	 (Premisa menor) \\Luego, todo hombre es sustancia	FigI:Barbara \\ \end{proof}
\begin{theorem}
todo tigre es sustancia
\label{th: 60}
\end{theorem}\begin{proof}\\todo vivo es sustancia	 (Premisa mayor) \\todo tigre es vivo	 (Premisa menor) \\Luego, todo tigre es sustancia	FigI:Barbara \\ \end{proof}
\begin{theorem}
todo reptil es sustancia
\label{th: 61}
\end{theorem}\begin{proof}\\todo vivo es sustancia	 (Premisa mayor) \\todo reptil es vivo	 (Premisa menor) \\Luego, todo reptil es sustancia	FigI:Barbara \\ \end{proof}
\begin{theorem}
algún tigre es sustancia
\label{th: 62}
\end{theorem}\begin{proof}\\todo vivo es sustancia	 (Premisa mayor) \\algún vivo es tigre	 (Premisa menor) \\Luego, algún tigre es sustancia	FigIII:Datisi \\ \end{proof}
\begin{theorem}
algún reptil es sustancia
\label{th: 63}
\end{theorem}\begin{proof}\\todo vivo es sustancia	 (Premisa mayor) \\algún vivo es reptil	 (Premisa menor) \\Luego, algún reptil es sustancia	FigIII:Datisi \\ \end{proof}
\begin{theorem}
algún animal no es Dios
\label{th: 64}
\end{theorem}\begin{proof}\\Dios no es mortal	 (Premisa mayor) \\algún mortal es animal	 (Premisa menor) \\Luego, algún animal no es Dios	FigIV:Fresison \\ \end{proof}
\begin{theorem}
algún sustancia no es Dios
\label{th: 65}
\end{theorem}\begin{proof}\\Dios no es mortal	 (Premisa mayor) \\algún mortal es sustancia	 (Premisa menor) \\Luego, algún sustancia no es Dios	FigIV:Fresison \\ \end{proof}
\begin{theorem}
todo animal no es Dios
\label{th: 66}
\end{theorem}\begin{proof}\\Dios no es mortal	 (Premisa mayor) \\todo animal es mortal	 (Premisa menor) \\Luego, todo animal no es Dios	FigII:Cesare \\ \end{proof}
\begin{theorem}
algún hombre es mortal
\label{th: 67}
\end{theorem}\begin{proof}\\todo vivo es mortal	 (Premisa mayor) \\algún vivo es hombre	 (Premisa menor) \\Luego, algún hombre es mortal	FigIII:Datisi \\ \end{proof}
\begin{theorem}
todo hombre es mortal
\label{th: 68}
\end{theorem}\begin{proof}\\todo vivo es mortal	 (Premisa mayor) \\todo hombre es vivo	 (Premisa menor) \\Luego, todo hombre es mortal	FigI:Barbara \\ \end{proof}
\begin{theorem}
todo tigre es mortal
\label{th: 69}
\end{theorem}\begin{proof}\\todo vivo es mortal	 (Premisa mayor) \\todo tigre es vivo	 (Premisa menor) \\Luego, todo tigre es mortal	FigI:Barbara \\ \end{proof}
\begin{theorem}
todo reptil es mortal
\label{th: 70}
\end{theorem}\begin{proof}\\todo vivo es mortal	 (Premisa mayor) \\todo reptil es vivo	 (Premisa menor) \\Luego, todo reptil es mortal	FigI:Barbara \\ \end{proof}
\begin{theorem}
algún tigre es mortal
\label{th: 71}
\end{theorem}\begin{proof}\\todo vivo es mortal	 (Premisa mayor) \\algún vivo es tigre	 (Premisa menor) \\Luego, algún tigre es mortal	FigIII:Datisi \\ \end{proof}
\begin{theorem}
algún reptil es mortal
\label{th: 72}
\end{theorem}\begin{proof}\\todo vivo es mortal	 (Premisa mayor) \\algún vivo es reptil	 (Premisa menor) \\Luego, algún reptil es mortal	FigIII:Datisi \\ \end{proof}
\begin{theorem}
algún sustancia es reptil
\label{th: 73}
\end{theorem}\begin{proof}\\todo reptil es animal	 (Premisa mayor) \\todo animal es sustancia	 (Premisa menor) \\Luego, algún sustancia es reptil	FigIV:Bramantip \\ \end{proof}
\begin{theorem}
algún mortal es reptil
\label{th: 74}
\end{theorem}\begin{proof}\\todo reptil es animal	 (Premisa mayor) \\todo animal es mortal	 (Premisa menor) \\Luego, algún mortal es reptil	FigIV:Bramantip \\ \end{proof}
\begin{theorem}
algún vivo es lagarto
\label{th: 75}
\end{theorem}\begin{proof}\\todo lagarto es reptil	 (Premisa mayor) \\todo reptil es vivo	 (Premisa menor) \\Luego, algún vivo es lagarto	FigIV:Bramantip \\ \end{proof}
\begin{theorem}
algún animal es reptil
\label{th: 76}
\end{theorem}\begin{proof}\\todo lagarto es reptil	 (Premisa mayor) \\todo lagarto es animal	 (Premisa menor) \\Luego, algún animal es reptil	FigIII:Darapti \\ \end{proof}
\begin{theorem}
algún vivo es cocodrilo
\label{th: 77}
\end{theorem}\begin{proof}\\todo cocodrilo es reptil	 (Premisa mayor) \\todo reptil es vivo	 (Premisa menor) \\Luego, algún vivo es cocodrilo	FigIV:Bramantip \\ \end{proof}
\begin{theorem}
algún vivo es rico
\label{th: 78}
\end{theorem}\begin{proof}\\todo hombre es rico	 (Premisa mayor) \\algún vivo es hombre	 (Premisa menor) \\Luego, algún vivo es rico	FigI:Darii \\ \end{proof}
\begin{theorem}
algún hombre es animal
\label{th: 79}
\end{theorem}\begin{proof}\\algún animal es Pedro	 (Premisa mayor) \\Pedro es hombre	 (Premisa menor) \\Luego, algún hombre es animal	FigIV:Dimaris \\ \end{proof}
\begin{theorem}
algún sustancia es Pedro
\label{th: 80}
\end{theorem}\begin{proof}\\algún animal es Pedro	 (Premisa mayor) \\todo animal es sustancia	 (Premisa menor) \\Luego, algún sustancia es Pedro	FigIII:Disamis \\ \end{proof}
\begin{theorem}
algún mortal es Pedro
\label{th: 81}
\end{theorem}\begin{proof}\\algún animal es Pedro	 (Premisa mayor) \\todo animal es mortal	 (Premisa menor) \\Luego, algún mortal es Pedro	FigIII:Disamis \\ \end{proof}
\begin{theorem}
algún hombre no es Juan
\label{th: 82}
\end{theorem}\begin{proof}\\Juan no es Pedro	 (Premisa mayor) \\Pedro es hombre	 (Premisa menor) \\Luego, algún hombre no es Juan	FigIV:Fesapo \\ \end{proof}
\begin{theorem}
algún hombre es rico
\label{th: 83}
\end{theorem}\begin{proof}\\algún rico es Pedro	 (Premisa mayor) \\Pedro es hombre	 (Premisa menor) \\Luego, algún hombre es rico	FigIV:Dimaris \\ \end{proof}
\begin{theorem}
algún animal es vivo
\label{th: 84}
\end{theorem}\begin{proof}\\algún vivo es hombre	 (Premisa mayor) \\todo hombre es animal	 (Premisa menor) \\Luego, algún animal es vivo	FigIV:Dimaris \\ \end{proof}
\begin{theorem}
algún sustancia es rico
\label{th: 85}
\end{theorem}\begin{proof}\\algún rico es animal	 (Premisa mayor) \\todo animal es sustancia	 (Premisa menor) \\Luego, algún sustancia es rico	FigIV:Dimaris \\ \end{proof}
\begin{theorem}
algún mortal es rico
\label{th: 86}
\end{theorem}\begin{proof}\\algún rico es animal	 (Premisa mayor) \\todo animal es mortal	 (Premisa menor) \\Luego, algún mortal es rico	FigIV:Dimaris \\ \end{proof}
\begin{theorem}
Dios no es hombre
\label{th: 87}
\end{theorem}\begin{proof}\\todo hombre es vivo	 (Premisa mayor) \\todo vivo no es Dios	 (Premisa menor) \\Luego, Dios no es hombre	FigIV:Camenes \\ \end{proof}
\begin{theorem}
Dios no es tigre
\label{th: 88}
\end{theorem}\begin{proof}\\todo tigre es vivo	 (Premisa mayor) \\todo vivo no es Dios	 (Premisa menor) \\Luego, Dios no es tigre	FigIV:Camenes \\ \end{proof}
\begin{theorem}
algún vivo es sustancia
\label{th: 89}
\end{theorem}\begin{proof}\\algún sustancia es animal	 (Premisa mayor) \\todo animal es vivo	 (Premisa menor) \\Luego, algún vivo es sustancia	FigIV:Dimaris \\ \end{proof}
\begin{theorem}
algún vivo es mortal
\label{th: 90}
\end{theorem}\begin{proof}\\algún mortal es animal	 (Premisa mayor) \\todo animal es vivo	 (Premisa menor) \\Luego, algún vivo es mortal	FigIV:Dimaris \\ \end{proof}
\begin{theorem}
Dios no es reptil
\label{th: 91}
\end{theorem}\begin{proof}\\todo reptil es vivo	 (Premisa mayor) \\todo vivo no es Dios	 (Premisa menor) \\Luego, Dios no es reptil	FigIV:Camenes \\ \end{proof}
\begin{theorem}
algún sustancia no es Juan
\label{th: 92}
\end{theorem}\begin{proof}\\algún animal no es Juan	 (Premisa mayor) \\todo animal es sustancia	 (Premisa menor) \\Luego, algún sustancia no es Juan	FigIII:Bocardo \\ \end{proof}
\begin{theorem}
algún mortal no es Juan
\label{th: 93}
\end{theorem}\begin{proof}\\algún animal no es Juan	 (Premisa mayor) \\todo animal es mortal	 (Premisa menor) \\Luego, algún mortal no es Juan	FigIII:Bocardo \\ \end{proof}
\begin{theorem}
Dios no es Pedro
\label{th: 94}
\end{theorem}\begin{proof}\\Pedro es hombre	 (Premisa mayor) \\Dios no es hombre	 (Premisa menor) \\Luego, Dios no es Pedro	FigII:Camestres \\ \end{proof}
\begin{theorem}
algún hombre es vivo
\label{th: 95}
\end{theorem}\begin{proof}\\todo animal es vivo	 (Premisa mayor) \\algún animal es hombre	 (Premisa menor) \\Luego, algún hombre es vivo	FigIII:Datisi \\ \end{proof}
\begin{theorem}
algún reptil es vivo
\label{th: 96}
\end{theorem}\begin{proof}\\todo animal es vivo	 (Premisa mayor) \\algún animal es reptil	 (Premisa menor) \\Luego, algún reptil es vivo	FigIII:Datisi \\ \end{proof}
\begin{theorem}
Pedro es sustancia
\label{th: 97}
\end{theorem}\begin{proof}\\todo vivo es sustancia	 (Premisa mayor) \\algún vivo es Pedro	 (Premisa menor) \\Luego, Pedro es sustancia	FigIII:Datisi \\ \end{proof}
\begin{theorem}
algún animal es sustancia
\label{th: 98}
\end{theorem}\begin{proof}\\todo vivo es sustancia	 (Premisa mayor) \\algún vivo es animal	 (Premisa menor) \\Luego, algún animal es sustancia	FigIII:Datisi \\ \end{proof}
\begin{theorem}
algún rico es sustancia
\label{th: 99}
\end{theorem}\begin{proof}\\todo vivo es sustancia	 (Premisa mayor) \\algún rico es vivo	 (Premisa menor) \\Luego, algún rico es sustancia	FigI:Darii \\ \end{proof}
\begin{theorem}
todo lagarto es sustancia
\label{th: 100}
\end{theorem}\begin{proof}\\todo vivo es sustancia	 (Premisa mayor) \\todo lagarto es vivo	 (Premisa menor) \\Luego, todo lagarto es sustancia	FigI:Barbara \\ \end{proof}
\begin{theorem}
todo cocodrilo es sustancia
\label{th: 101}
\end{theorem}\begin{proof}\\todo vivo es sustancia	 (Premisa mayor) \\todo cocodrilo es vivo	 (Premisa menor) \\Luego, todo cocodrilo es sustancia	FigI:Barbara \\ \end{proof}
\begin{theorem}
algún lagarto es sustancia
\label{th: 102}
\end{theorem}\begin{proof}\\todo vivo es sustancia	 (Premisa mayor) \\algún lagarto es vivo	 (Premisa menor) \\Luego, algún lagarto es sustancia	FigI:Darii \\ \end{proof}
\begin{theorem}
algún cocodrilo es sustancia
\label{th: 103}
\end{theorem}\begin{proof}\\todo vivo es sustancia	 (Premisa mayor) \\algún cocodrilo es vivo	 (Premisa menor) \\Luego, algún cocodrilo es sustancia	FigI:Darii \\ \end{proof}
\begin{theorem}
algún hombre no es Dios
\label{th: 104}
\end{theorem}\begin{proof}\\Dios no es mortal	 (Premisa mayor) \\algún mortal es hombre	 (Premisa menor) \\Luego, algún hombre no es Dios	FigIV:Fresison \\ \end{proof}
\begin{theorem}
algún vivo no es Dios
\label{th: 105}
\end{theorem}\begin{proof}\\Dios no es mortal	 (Premisa mayor) \\algún mortal es vivo	 (Premisa menor) \\Luego, algún vivo no es Dios	FigIV:Fresison \\ \end{proof}
\begin{theorem}
algún tigre no es Dios
\label{th: 106}
\end{theorem}\begin{proof}\\Dios no es mortal	 (Premisa mayor) \\algún mortal es tigre	 (Premisa menor) \\Luego, algún tigre no es Dios	FigIV:Fresison \\ \end{proof}
\begin{theorem}
todo hombre no es Dios
\label{th: 107}
\end{theorem}\begin{proof}\\Dios no es mortal	 (Premisa mayor) \\todo hombre es mortal	 (Premisa menor) \\Luego, todo hombre no es Dios	FigII:Cesare \\ \end{proof}
\begin{theorem}
todo tigre no es Dios
\label{th: 108}
\end{theorem}\begin{proof}\\Dios no es mortal	 (Premisa mayor) \\todo tigre es mortal	 (Premisa menor) \\Luego, todo tigre no es Dios	FigII:Cesare \\ \end{proof}
\begin{theorem}
todo reptil no es Dios
\label{th: 109}
\end{theorem}\begin{proof}\\Dios no es mortal	 (Premisa mayor) \\todo reptil es mortal	 (Premisa menor) \\Luego, todo reptil no es Dios	FigII:Cesare \\ \end{proof}
\begin{theorem}
algún reptil no es Dios
\label{th: 110}
\end{theorem}\begin{proof}\\Dios no es mortal	 (Premisa mayor) \\algún reptil es mortal	 (Premisa menor) \\Luego, algún reptil no es Dios	FigII:Festino \\ \end{proof}
\begin{theorem}
Pedro no es Dios
\label{th: 111}
\end{theorem}\begin{proof}\\Dios no es mortal	 (Premisa mayor) \\algún mortal es Pedro	 (Premisa menor) \\Luego, Pedro no es Dios	FigIV:Fresison \\ \end{proof}
\begin{theorem}
algún rico no es Dios
\label{th: 112}
\end{theorem}\begin{proof}\\Dios no es mortal	 (Premisa mayor) \\algún mortal es rico	 (Premisa menor) \\Luego, algún rico no es Dios	FigIV:Fresison \\ \end{proof}
\begin{theorem}
Pedro es mortal
\label{th: 113}
\end{theorem}\begin{proof}\\todo vivo es mortal	 (Premisa mayor) \\algún vivo es Pedro	 (Premisa menor) \\Luego, Pedro es mortal	FigIII:Datisi \\ \end{proof}
\begin{theorem}
algún animal es mortal
\label{th: 114}
\end{theorem}\begin{proof}\\todo vivo es mortal	 (Premisa mayor) \\algún vivo es animal	 (Premisa menor) \\Luego, algún animal es mortal	FigIII:Datisi \\ \end{proof}
\begin{theorem}
algún rico es mortal
\label{th: 115}
\end{theorem}\begin{proof}\\todo vivo es mortal	 (Premisa mayor) \\algún rico es vivo	 (Premisa menor) \\Luego, algún rico es mortal	FigI:Darii \\ \end{proof}
\begin{theorem}
todo lagarto es mortal
\label{th: 116}
\end{theorem}\begin{proof}\\todo vivo es mortal	 (Premisa mayor) \\todo lagarto es vivo	 (Premisa menor) \\Luego, todo lagarto es mortal	FigI:Barbara \\ \end{proof}
\begin{theorem}
todo cocodrilo es mortal
\label{th: 117}
\end{theorem}\begin{proof}\\todo vivo es mortal	 (Premisa mayor) \\todo cocodrilo es vivo	 (Premisa menor) \\Luego, todo cocodrilo es mortal	FigI:Barbara \\ \end{proof}
\begin{theorem}
algún lagarto es mortal
\label{th: 118}
\end{theorem}\begin{proof}\\todo vivo es mortal	 (Premisa mayor) \\algún lagarto es vivo	 (Premisa menor) \\Luego, algún lagarto es mortal	FigI:Darii \\ \end{proof}
\begin{theorem}
algún cocodrilo es mortal
\label{th: 119}
\end{theorem}\begin{proof}\\todo vivo es mortal	 (Premisa mayor) \\algún cocodrilo es vivo	 (Premisa menor) \\Luego, algún cocodrilo es mortal	FigI:Darii \\ \end{proof}
\begin{theorem}
algún sustancia es lagarto
\label{th: 120}
\end{theorem}\begin{proof}\\todo lagarto es reptil	 (Premisa mayor) \\todo reptil es sustancia	 (Premisa menor) \\Luego, algún sustancia es lagarto	FigIV:Bramantip \\ \end{proof}
\begin{theorem}
algún mortal es lagarto
\label{th: 121}
\end{theorem}\begin{proof}\\todo lagarto es reptil	 (Premisa mayor) \\todo reptil es mortal	 (Premisa menor) \\Luego, algún mortal es lagarto	FigIV:Bramantip \\ \end{proof}
\begin{theorem}
Dios no es lagarto
\label{th: 122}
\end{theorem}\begin{proof}\\todo lagarto es reptil	 (Premisa mayor) \\Dios no es reptil	 (Premisa menor) \\Luego, Dios no es lagarto	FigII:Camestres \\ \end{proof}
\begin{theorem}
algún sustancia es cocodrilo
\label{th: 123}
\end{theorem}\begin{proof}\\todo cocodrilo es reptil	 (Premisa mayor) \\todo reptil es sustancia	 (Premisa menor) \\Luego, algún sustancia es cocodrilo	FigIV:Bramantip \\ \end{proof}
\begin{theorem}
algún mortal es cocodrilo
\label{th: 124}
\end{theorem}\begin{proof}\\todo cocodrilo es reptil	 (Premisa mayor) \\todo reptil es mortal	 (Premisa menor) \\Luego, algún mortal es cocodrilo	FigIV:Bramantip \\ \end{proof}
\begin{theorem}
todo lagarto no es Dios
\label{th: 125}
\end{theorem}\begin{proof}\\Dios no es mortal	 (Premisa mayor) \\todo lagarto es mortal	 (Premisa menor) \\Luego, todo lagarto no es Dios	FigII:Cesare \\ \end{proof}
\begin{theorem}
todo cocodrilo no es Dios
\label{th: 126}
\end{theorem}\begin{proof}\\Dios no es mortal	 (Premisa mayor) \\todo cocodrilo es mortal	 (Premisa menor) \\Luego, todo cocodrilo no es Dios	FigII:Cesare \\ \end{proof}
\begin{theorem}
algún lagarto no es Dios
\label{th: 127}
\end{theorem}\begin{proof}\\Dios no es mortal	 (Premisa mayor) \\algún lagarto es mortal	 (Premisa menor) \\Luego, algún lagarto no es Dios	FigII:Festino \\ \end{proof}
\begin{theorem}
algún cocodrilo no es Dios
\label{th: 128}
\end{theorem}\begin{proof}\\Dios no es mortal	 (Premisa mayor) \\algún cocodrilo es mortal	 (Premisa menor) \\Luego, algún cocodrilo no es Dios	FigII:Festino \\ \end{proof}
\\\small{Este libro fue generado automáticamene por NOVA.} \\
\small{Salvador D. Escobedo, Julio 2016}.
\end{document}